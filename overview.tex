%\frameT{Research on Preferences}{
%	Q: How do we represent preferences over combinatorial domains?
%
%	\begin{enumerate}
%		\item Quantitative:
%			\begin{enumerate}
%				\item Utility/Cost Functions
%				\item Possibilistic Logic\footcitefull{DuboisLP91}
%				\item Fuzzy Preference Relations\footcitefull{orlovsky1978decision}
%				\item Penalty Logic\footcitefull{pinkas1991propositional}
%			\end{enumerate}
%		\seti
%	\end{enumerate}
%}

%\frameT{Preference Modeling}{
%	Q: How do we compactly represent qualitative preferences over combinatorial domains?
%
%	\begin{enumerate}
%		\item Answer-Set Optimization Theories\footcitefull{Brewka:ASO}
%		\item Ceteris Paribus Networks (e.g., CP-nets\footcitefull{bbdh03},
%					TCP-nets\footcitefull{BrafmanD02:TCP},
%					CI-nets\footcitefull{bouveret2009conditional})
%		\item Conditional Preference Theories\footcitefull{Wilson04extendingcp-nets}
%		\seti
%	\end{enumerate}
%}

\frameT{Preference Modeling}{
	Q: How do we compactly represent qualitative preferences over combinatorial domains?

	\begin{enumerate}
		\item Preference Trees (P-trees)\footcitefull{fraser1994}\textsuperscript{,}\footcitefull{wsh/mpref14/LiuT}
		\item Partial Lexicographic Preference Trees (PLP-trees)\footcitefull{conf/aaai15/LiuT}
		\item Lexicographic Preference Trees (LP-trees)\footcitefull{booth:learningLP}\textsuperscript{,}\footcitefull{conf/adt13/LiuT}
	\end{enumerate}
}

\frameT{Preference Learning}{
	Q: How do we learn predictive qualitative preference models over combinatorial domains?

	\begin{enumerate}
		%\item Ceteris Paribus Networks (e.g., 
		%			CP-nets\footcitefull{lang2009complexity}\textsuperscript{,}\footcitefull{koriche2010learning}\textsuperscript{,}\footcitefull{chevaleyre2011learning})
		%\item Preference Trees (e.g., LP-trees\footcitefull{booth:learningLP}, 
		%			CLP-trees\footcitefull{brauning2012learning}, \tbf{PLP-trees}\footcitefull{conf/aaai15/LiuT})
		\item Partial Lexicographic Preference Trees 
					(PLP-trees)\footcitefull{schmitt2006complexity}\textsuperscript{,}\footcitefull{dombi2007learning}\textsuperscript{,}\footcitefull{conf/aaai15/LiuT}
			\begin{itemize}
				\item Active and passive learning
				\item Compute a (possibly small) PLP-tree consistent with all the data
				\item Compute a PLP-tree that agrees with the data as much as possible
			\end{itemize}
		\item Preference Forests\footcitefull{confInPrep/ijcai16/LiuT1}
		\item Preference Approximation\footcitefull{confInPrep/X/CPN_LPT}
	\end{enumerate}
}

\frameT{Preference Reasoning}{
	Q: How do we reason about preferences over combinatorial domains?

	\begin{enumerate}
		\item Preference Optimization\footcitefull{lang:aggLP}\textsuperscript{,}\footcitefull{conf/adt13/LiuT}\textsuperscript{,}\footcitefull{wsh/mpref14/LiuT}\textsuperscript{,}\footcitefull{conf/adt15/liuT}:
			\begin{itemize}
				\item Dominance testing: $o_1 \succeq_P o_2$?
				\item Optimality testing: $o_1 \succeq_P o_2$ for all $o_2 \not = o_1$?
				\item Optimality computing: what is the optimal object wrt $P$?
				\item Preference aggregation: which candidate wins the election?
			\end{itemize}
		\item Preference Misrepresentation\footcitefull{brandt2012computational}\textsuperscript{,}\footcitefull{confInPrep/X/LiuT}:
			\begin{itemize}
				%\item Control
				\item Manipulation
				%\item Bribery
			\end{itemize}
	\end{enumerate}
}

\frameT{Preference Applications}{
	Q: What fields can we apply preferences to?

	\begin{enumerate}
		\item Role-playing Games:
			\begin{itemize}
				\item Hedonic games\footcitefull{conf/adt13/Spradling}
			\end{itemize}
		\item Automated Planning and Scheduling:
			\begin{itemize}
				\item Trip planning\footcitefull{abs/parc/Liu1}
			\end{itemize}
		\item Data-Driven Decision Making:
			\begin{itemize}
				\item Predictive models\footcitefull{confInPrep/ijcai16/LiuT2}
			\end{itemize}
	\end{enumerate}
}

\frameT{Outline}{
	\begin{enumerate}
		\item The languages of PLP-trees and LP-trees
		\item Learning preference models in case of PLP-trees
		\item Reasoning with preferences:
			\begin{itemize}
				\item Computing winners and ``strong" candidates when votes are LP-trees
				\item Application in trip planning
			\end{itemize}
		\item Future research directions
	\end{enumerate}
}
